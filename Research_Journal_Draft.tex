% Please do not change the document class
\documentclass{scrartcl}

% Please do not change these packages
\usepackage[hidelinks]{hyperref}
\usepackage[none]{hyphenat}
\usepackage{setspace}
\doublespace

% You may add additional packages here
\usepackage{amsmath}

\title{Research Journal: Principles of Computing}

\subtitle{COMP110 - Research Journal}

\author{1608351}

\begin{document}

\maketitle

\section{Computational Philosophy}

One of the big debates in computing, is how to define a computer. Is, as has been debated, everything a computer? The answer as far as Horsman et al, are concerned, is no; not everything in the universe is acting as a computer. They argue that declaring everything as a computer, negates the existence of physical computation in its own right and that a framework is needed for determining whether a physical system is performing a computation. As such, they propose, that a computer can be defined if all the following are present:
- A strong physical theory of the computer, which we are confident with and have tested.
- A computational entity.
- A representation relation.
- The ability to encode and decode information.
- One or more fundamental dynamic operations (such as logic gates) which are able to transfer input states to output states. 
- Finally, all these elements need to pass the relevant commuting diagrams proposed. \cite{Horsman2014}. 


\section{Best Practices in Programming}

Computing is a fast-changing field, and as such, it is necessary to regularly review best practices to ensure they are still relevant and effective. 

One such practice that has been subject to debate, is the use of flowcharts in computer programming. It seems that, as in Mayer's study, whilst flowcharts may be of help with high-level composition, they do not enhance an individuals comprehension \cite{Mayer}. Therefore, it may be possible for individuals to write the program and ones similar, but transferring their knowledge to other coding tasks will be hindered by a lack of understanding at a lower level. 

Whilst widely used in both professional and educational programming environments, flowcharts have been found by some researchers to have little improvement on programming skills. In all five of the experiments performed by B. Sneiderman et al, testing the composition, comprehension, debugging and/or modification skills in university students learning programming, found no significant positive or negative advantages in the use of flowcharts \cite{Sneiderman}. In one sample, the non-flowchart group outperformed those using flowcharts. It was also observed, in experiment 2, that the subjects rarely made use of the flowcharts, instead working with the program alone. For example, Weinberg explains \textit{`we find no evidence that the original coding plus flow diagrams is any easier to understand than the original coding itself'}\cite{Weinberg}. 

Yet in opposition to such findings, flowcharts have been found effective in other sectors by helping to separate relevant information from the irrelevant \cite{Kammann1975}. Further still, research has found evidence to support the use of flowcharts over pseudocode in understanding algorithms, \textit{‘the more complex the algorithm, the more beneficial structured flowcharts are.’} \cite{Scanlan}. As recommended by Sneiderman, performing further study of how structured flowcharts aid, or indeed do not aid, computing professionals may help to determine how relevant the use of flowcharts are to the industry.

A second example of programming practices being subject to review, is the Go To Statment. In 1968, Dijkstra wrote an influential argument against the use of the Go To Statment in all high-level programming languages. 

\section{Development Practices}

Following on from best programming practices, in the game development industry improving development practices are key to ensuring the efficent and succesful delivery of working software to clients. Game and software development teams are increasingly adopting the Agile Manifesto, with it's focus on self-organising teams, regular reflection, and communication \cite{Agile}. Yet, as is to be expected, there are challenges to overcome in the adoption of agile. Research has found certain personality types to have a predisposition towards agile practices, namely extraverted and open individuals\cite{BishopDeokar}. However, those with more introverted and neurotic traits are reportdely less inclinded towards the strong focus on accounatbility and communication found in agile. Such findings, suggest a need to consider personality traits, and the balancing of such traits in teams ,when implementing team-driven programming projects.


\section{The Future of Computing}


\section{Conclusion}


\bibliographystyle{ieeetran}

\bibliography{journal_references}