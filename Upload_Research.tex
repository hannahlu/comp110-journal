% Please do not change the document class
\documentclass{scrartcl}

% Please do not change these packages
\usepackage[hidelinks]{hyperref}
\usepackage[none]{hyphenat}
\usepackage{setspace}
\doublespace

% You may add additional packages here
\usepackage{amsmath}

% Title 
\title{The History of Computing}

\subtitle{COMP110 - Research Journal}

\author{1608351}

\begin{document}

One of the big debates in computing, is how to define a computer. Is, as has been argued, everything a computer? The answer as far as Horsman C et al, are concerned, is no; not everything in the universe is acting as a computer. They argue that declaring everything as a computer, negates the existence of physical computation in its own right and that a framework is needed for determining whether a physical system is performing a computation \cite{Horsman2014}.

\section{Programming Practices}
Computing is a fast-changing field, and as such, it is necessary to regularly review best practices to ensure they are still relevant and effective. One such practice that has been subject to debate, is the use of flowcharts in computer programming. Whilst widely used in both professional and educational programming environments, flowcharts have been found by some researchers to have little improvement on programming skills. Whilst widely used in both professional and educational programming environments, flowcharts have been found by some researchers to have little improvement on programming skills. For example, Weinberg explains \textit{`we find no evidence that the original coding plus flow diagrams is any easier to understand than the original coding itself'}\cite{Weinberg}. However, other researchers, such as Kammann, have found the use of flowcharts to be effective in other sectors, as they helped to separate relevant information from the irrelevant \cite{Kammann1975}.

\section{Development Practices}


\section{The Future of Computing}
 

\section{Conclusion}



\bibliographystyle{ieeetran}
\bibliography{journal_references}

\end{document}